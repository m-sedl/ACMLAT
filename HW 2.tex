\documentclass{article}

\usepackage[russian]{babel}
\usepackage[letterpaper,top=2cm,bottom=2cm,left=3cm,right=3cm,marginparwidth=1.75cm]{geometry}
\usepackage{amsmath}
\usepackage{amssymb}
\usepackage{stmaryrd}
\usepackage{bussproofs}

\title{Доп. главы матлогики. ДЗ 2}
\author{Седлярский Михаил Андреевич 21.М07-мм}
\date{ }

\setlength{\parindent}{0em}

\begin{document}
\maketitle

\section*{Задание 1}
Если \(T \cup U \models \bot\), то верно следующее утверждение: 
\(T \models \neg U\).

Согласно интерполяционной теореме Крейга есть такая формула \(\theta\), что
\(T \models \theta\), \(\theta \models \neg U\) и 
\(atoms(\theta) \subseteq atoms(T) \cap atoms(\neg U)\). 
Т.к. \(\theta \models \neg U\), то справедливо и \(U \models \neg \theta\).
В силу того, что \(\neg \theta \lor \neg U\) равносильно \(\neg U \lor \neg \theta\)

Т.е. для таких множеств формул \(T \cup U \models \bot\) есть формула \(\theta\), что
\(T \models \theta\), \(U \models \neg \theta\) 
и \(atoms(\theta) \subseteq atoms(T) \cap atoms(U)\).


\section*{Задание 2}


\end{document}