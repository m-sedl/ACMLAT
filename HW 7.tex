\documentclass{article}

\usepackage[russian]{babel}
\usepackage[letterpaper,top=2cm,bottom=2cm,left=3cm,right=3cm,marginparwidth=1.75cm]{geometry}
\usepackage{amsmath}
\usepackage{amssymb}
\usepackage{stmaryrd}
\usepackage{bussproofs}
\usepackage{hyperref}
\usepackage[usenames]{color}
\usepackage{colortbl}

\title{Доп. главы матлогики. ДЗ 7}
\author{Седлярский Михаил Андреевич 21.М07-мм}
\date{ }

\setlength{\parindent}{0em}

\begin{document}
\maketitle

\section*{Задание 1}
Записать высказывания на языке первого порядка
\begin{enumerate}
    \item 
        \(\forall x, y: (coauthors(x, y) \iff \exists p : author(x, p) \land author(y, p))\)
    \item 
        \(\exists x: researcher(x) \land (\forall p: author(x, p) \rightarrow \exists y : researcher(y) \land author(y, p) \land \neg(x = y))\)
    \item 
        \(\forall x, y: coauthors(x, y) \rightarrow \neg (\exists p: author(x, p) \land reviewer(y, p))\)
    \item 
        \(\forall x \exists y: researcher(x) \land (tutor(x) = y) \land coauthors(x, y)\)
\end{enumerate}

\section*{Задание 2}
Истинны ли следующие формулы в структуре M при оценке v?
\begin{enumerate}
    \item 
        \( p(x,y) \rightarrow (\forall z \exists w : q(z, f(w))) \)

        Ложь в структуре M при оценке v т.к. при \(z \mapsto \star\) будет ложным выражение \(q(\star, f(w))\)
    \item 
        \( p(x,y) \rightarrow (\exists z \exists w : q(z, w)) \)

        Заменим свободные переменные согласно оценке из задания.
        \[ p(\star,\circ) \rightarrow (\exists z \exists w : q(z, w)) \]
        \[ \top \rightarrow (\exists z \exists w : q(z, w)) \]
        Допустим, что \(z \mapsto \circ w \mapsto \circ\) при которых \(q(z, w)\) выполнима.
        Следовательно, исходная формула истинна в структуре M при оценке v.
    \item 
        \(\forall x \exists y : ((p(x,y) \land q(y,x)) \rightarrow p(y,x))\)

        Допустим, что \(x \mapsto \circ\). 

        Тогда получаем: \(\exists y : ((p(\circ,y) \land q(y,\circ)) \rightarrow p(y,\circ))\).
        Допустим \(y \mapsto \circ\). Получаем \((\bot \land \top) \rightarrow \bot\) т.е. частный случай истинный.
        
        Допустим, что \(x \mapsto \star\). 

        Тогда получаем: \(\exists y : ((p(\star,y) \land q(y,\star)) \rightarrow p(y,\star))\).
        Допустим \(y \mapsto \star\). Получаем \((\bot \land \bot) \rightarrow \bot\) т.е. частный случай истинный.

        Следовательно, исходная формула истинна в структуре M при оценке v.
\end{enumerate}

\section*{Задание 3}
\begin{enumerate}
    \item 
        \(\exists x : (p(x) \lor q(x)) \leftrightarrow (\exists x : p(x)) \lor (\exists x : q(x))\)
        
        По правилам интерпретации формул логики первого порядка получаем:

        \(
            M \models_v \exists x : (p(x) \lor q(x)) \leftrightarrow
            M \models_{v[x \mapsto a]} p(x)
        \) или \(M \models_{v[x \mapsto a]} q(x)\)

        Очевидно, что независимо от структуры M левая часть выполняется только 
        тогда и только тогда, когда выполняется хотя бы одна из правых формул.
    \item 
        \(
            (\forall x \forall y : (p(x,y) \rightarrow p(y,x)))
            \rightarrow \forall z : p(z,z)
        \)
        Допустим, что на носителе структуры M заданы отношения симметрии и рефлексии.
        Тогда исходная формула выполнима независимо от оценки.

        Очевидно, что упомянутые отношения могут быть не заданы и тогда формула не будет выполняться.
        Следовательно, исходное предложение выполнимо, но не общезначимо.

        Например, невыполнимое в структуре \(M\) такой, что:

            \(|M| = \{1, 2\}\)

            \(M(p) = \{<1,2>, <2,1>\}\)

            \(ar = \{p \mapsto 2\}\)

            \(M(f) = \emptyset\)
    \item 
        \(
            (\forall x : (\neg p(x) \lor \neg q(x))) 
            \leftrightarrow (\neg \forall x : (p(x) \land q(x)))
        \)

        Раскроем отрицание правой части. Получаем:

        \(
            (\forall x : (\neg p(x) \lor \neg q(x))) 
            \leftrightarrow (\exists x : (\neg p(x) \lor \neg q(x)))
        \)

        Это точно выполняется в структуре, в носителе которой, всего один элемент. 
        Но может не выполняться в других структурах. Следовательно, предложение выполнимо, но не общезначимою
    \item 
        \(\exists y \forall x : p(y, x) \land \exists x \forall y : \neg p(y, x)\)
        
        Переименуем переменные для удобства:

        \(\exists y \forall x : p(y, x) \land \exists z \forall u : \neg p(u, z)\)

        Левая часть выражения утверждает, что в множестве \(M(p)\) существует такой \(y\), 
        входящий в пару со всеми возможными значениями из носителя. 
        Правая часть утверждает, что можно найти такой \(z\), 
        который вообще не встречается во множестве пар \(M(p)\).

        Противоречие. Следовательно, предложение невыполнимое.
\end{enumerate}

\section*{Задание 4}
\section*{Задание 5}
Пусть \(\Sigma^{Gr}_f = \{\cdot, ^{-1}\}\)

Пусть \(\Gamma = \{\)
    \[\forall x, y : (x \cdot y = y \land y \cdot x = y) \rightarrow (x = y \cdot y^{-1} \land x = y^{-1} \cdot y)\]
    \[\forall x, y, z: (x \cdot y) \cdot z = x \cdot (y \cdot z)\]
\(\}\)
\end{document}