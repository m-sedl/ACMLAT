\documentclass{article}

\usepackage[russian]{babel}
\usepackage[letterpaper,top=2cm,bottom=2cm,left=3cm,right=3cm,marginparwidth=1.75cm]{geometry}
\usepackage{amsmath}
\usepackage{amssymb}
\usepackage{stmaryrd}
\usepackage{bussproofs}
\usepackage{hyperref}
\usepackage[usenames]{color}
\usepackage{colortbl}

\title{Доп. главы матлогики. ДЗ 3}
\author{Седлярский Михаил Андреевич 21.М07-мм}
\date{ }

\setlength{\parindent}{0em}

\begin{document}
\maketitle

\section*{Задание 1}
Приведём формулу \(\neg(\neg(p \land q) \rightarrow \neg r)\) в КНФ с помощью преобразования Цейтина.
Запишем формулу в каноничном виде: \(\neg((p \land q) \lor \neg r)\).

Положим \(\Delta = \emptyset\).

Начнём преобразование с самой внутренней подформулы: \(p \land q\).
Левая и правая её части - проп.переменные, поэтому преобразование Цейтина
над ними никак не  изменит наш аккумулятор.

Введём новую переменную \(a_1\). Получаем: \(
    \Delta = \{
        \neg a_1 \lor p,
        \neg a_1 \lor q,
        \neg p \lor \neg q \lor a_1 
    \}
\)

Затем поднимемся на уровень дизъюнкции и введём переменную \(a_2\).
Получаем следующее множество: \(
    \Delta = \{
        \neg a_2 \lor a_1 \lor \neg r,
        \neg a_1 \lor a_2,
        r \lor a_2,
        \neg a_1 \lor p,
        \neg a_1 \lor q,
        \neg p \lor \neg q \lor a_1 
    \}
\)

На уровне отрицания добавляем отрицание \(a_2\). 
В итоге получаем:
\[
    \neg a_2
    \land  (\neg a_2 \lor a_1 \lor \neg r)
    \land (\neg a_1 \lor a_2)
    \land (r \lor a_2)
    \land (\neg a_1 \lor p)
    \land (\neg a_1 \lor q)
    \land (\neg p \lor \neg q \lor a_1)
\]

\section*{Задание 2}
Докажем корректность преобразования Цейтина. 
Т.е. если формуле F соответствует F`, то они эквивыполнимы.

Воспользуемся индукцией по построению формулы.

База с проп. переменной.
\((l \leftrightarrow p) \land l \models p\)
Упростим выражение: \(l \land p \models p\). 
Очевидно, что любая модель левой части является моделью правой.

Переход.

1) \((l \leftrightarrow \phi) \land \neg l \models \neg \phi\)
Упростим выражение: \(\neg l \land \neg \phi \models \neg \phi\).
Случай аналогичен предыдущему.

2) Пусть \(\phi\) - функция, не содержащая в себе функции \(\psi\).

\[(l \leftrightarrow \psi) \land (\phi \land l) \models \phi \land \psi\]
\[(\phi \land l) \land (l \land \psi) \lor (\neg l \land \neg \psi) \land (\phi \land l) \models \phi \land \psi\]
\[\phi \land \psi \models \phi \land \psi\]

3) Пусть \(\phi\) - функция, не содержащая в себе функции \(\psi\).
\[(l \leftrightarrow \psi) \land (\phi \lor l) \models \phi \lor \psi\]
\[(\phi \lor l) \land (l \land \psi) \lor (\neg l \land \neg \psi) \land (\phi \lor l) \models \phi \lor \psi\]
\[(l \land \psi) \lor (\neg l \land \neg \psi \land \phi) \models \phi \lor \psi\]
Левое выражение верно если \(M\llbracket \psi \rrbracket = 1\) либо \(M\llbracket \phi \rrbracket = 1\). 
В обоих случаях будет выполняться \(\phi \lor \psi\). 

Ч.т.д.

\section*{Задание 3}
\url{https://github.com/MihailITPlace/acmlat-3hw-3-problem}

\section*{Задание 4}
\url{https://github.com/MihailITPlace/acmlat-3hw-4-problem}

\section*{Задание 5}
Запишем формулу в виду множества дизъюнктов.
\[p \lor q \lor r\]
\[\neg p \lor \neg q \lor \neg r\]
\[\neg p \lor q \lor r\]
\[\neg q \lor r\]
\[q \lor \neg r\]

Допустим \(M(q) = 1\). Получаем:
\[\textcolor{green}{p \lor q \lor r}\]
\[\neg p \lor \textcolor{red}{\neg q} \lor \neg r\]
\[\textcolor{green}{\neg p \lor q \lor r}\]
\[\textcolor{red}{\neg q} \lor r\]
\[\textcolor{green}{q \lor \neg r}\]

Допустим \(M(r) = 1\). Получаем:
\[\textcolor{green}{p \lor q \lor r}\]
\[\neg p \lor \textcolor{red}{\neg q} \lor \textcolor{red}{\neg r}\]
\[\textcolor{green}{\neg p \lor q \lor r}\]
\[\textcolor{red}{\neg q} \lor \textcolor{green}{r}\]
\[\textcolor{green}{q \lor \neg r}\]

Допустим \(M(p) = 0\). Получаем:
\[\textcolor{green}{p \lor q \lor r}\]
\[\textcolor{green}{\neg p} \lor \textcolor{red}{\neg q} \lor \textcolor{red}{\neg r}\]
\[\textcolor{green}{\neg p \lor q \lor r}\]
\[\textcolor{red}{\neg q} \lor \textcolor{green}{r}\]
\[\textcolor{green}{q \lor \neg r}\]

Множество всех дизъюнктов пусто, итоговая модель: 
\(M[q \mapsto 1, r \mapsto 1, p \mapsto 0]\)
\end{document}