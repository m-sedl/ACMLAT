\documentclass{article}

\usepackage[russian]{babel}
\usepackage[letterpaper,top=2cm,bottom=2cm,left=3cm,right=3cm,marginparwidth=1.75cm]{geometry}
\usepackage{amsmath}
\usepackage{stmaryrd}

\title{Доп. главы матлогики. ДЗ 1}
\author{Седлярский Михаил Андреевич 21.М07-мм}
\date{ }

\setlength{\parindent}{0em}

\begin{document}
\maketitle

\section*{Задание 1}
\begin{enumerate}
    \item
        Доказать, что \(\models \phi \rightarrow \psi\) тогда и только тогда, когда \(\phi \models{\psi}\).

        По определению \(\phi \rightarrow \psi\) -- краткая запись \(\neg \phi \lor \psi\).
        Если \(\phi \models{\psi}\), тогда если \(M_i\llbracket \phi \rrbracket = 1\), то и \(M_i\llbracket \psi \rrbracket = 1\).

        Рассмотрим два случая.
        \begin{enumerate}
            \item
                \(M_i\llbracket \phi \rrbracket = 0\).
                Получаем \(M_i\llbracket \neg 0 \lor \psi \rrbracket = MAX(1, M_i\llbracket \psi \rrbracket) = 1\)
            \item
                \(M_i\llbracket \phi \rrbracket = 1\).
                Тогда получаем \(M_i\llbracket \neg 1 \lor \psi \rrbracket = MAX(0, M_i\llbracket \psi \rrbracket) = MAX(0, 1) = 1\)
        \end{enumerate}

        Во второй интерпретации формула будет выполняться только при условии \(\phi \models \psi\), ч.т.д.
    \item
        Доказать, что \(\models \phi \leftrightarrow \psi\) тогда и только тогда, когда \(\phi \sim \psi\).

        По определению \(\phi \leftrightarrow \psi\) -- краткая запись \((\phi \land \psi) \lor (\neg \phi \land \neg \psi)\).
        Если \(\phi \sim \psi\), тогда если \(M_i\llbracket \phi \rrbracket = 1\),
        то \(
        M_i\llbracket \psi \rrbracket = 1\) и если
        \(M_j\llbracket \psi \rrbracket = 1\), то
        \(M_j\llbracket \phi \rrbracket = 1
        \)

        Рассмотрим два случая.
        \begin{enumerate}
            \item
                \(M_i\llbracket \phi \rrbracket = 0\). Получаем:
                \[
                    M_i\llbracket (\phi \land \psi) \lor (\neg \phi \land \neg \psi) \rrbracket
                    = MAX(MIN(0, M_i\llbracket\psi\rrbracket), MIN(1, 1))
                    = MAX(0, 1)
                    = 1
                \]
                Мы принимаем \(M_i\llbracket\neg\psi\rrbracket\) равным 1 т.к. если бы \(M_i\llbracket\psi\rrbracket = 1\), то и \(M_i\llbracket\phi\rrbracket = 1\), что противоречит условию рассматриваемого случая.
            \item
                \(M_i\llbracket \phi \rrbracket = 1\). Получаем:
                \[
                    M_i\llbracket (\phi \land \psi) \lor (\neg \phi \land \neg \psi) \rrbracket
                    = MAX(MIN(1, 1), MIN(0, 0))
                    = MAX(1, 0)
                    = 1
                \]
        \end{enumerate}

        Первый случай выполняется только при \(\psi \models \phi\).
        В то время как второй случай выполняется в силу \(\phi \models \psi\).
        Т.е. оба случая выпонляются тогда и только тогда, когда \(\phi \sim \psi\), ч.т.д.
\end{enumerate}

\section*{Задание 2}
\begin{enumerate}
    \item
        Докажем \(\neg \neg \phi \sim \phi\). Данная запись означает, что значения формул \(\neg \neg \phi\) и \(\phi\) совпадают при любой интерпретации т.е. \(M\llbracket \neg \neg \phi\rrbracket = M\llbracket\phi\rrbracket\).

        Упростим левую часть выражения:
        \[
            M\llbracket\neg \neg \phi \rrbracket
            = 1 - M\llbracket \neg \phi \rrbracket
            = 1 - (1 - M\llbracket\phi\rrbracket)
            = M\llbracket\phi\rrbracket
        \]
        Ч.т.д.
\end{enumerate}
\end{document}