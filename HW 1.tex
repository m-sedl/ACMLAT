\documentclass{article}

\usepackage[russian]{babel}
\usepackage[letterpaper,top=2cm,bottom=2cm,left=3cm,right=3cm,marginparwidth=1.75cm]{geometry}
\usepackage{amsmath}
\usepackage{amssymb}
\usepackage{stmaryrd}
\usepackage{bussproofs}

\title{Доп. главы матлогики. ДЗ 1}
\author{Седлярский Михаил Андреевич 21.М07-мм}
\date{ }

\setlength{\parindent}{0em}

\begin{document}
\maketitle

\section*{Задание 1}
\begin{enumerate}
    \item
        Доказать, что \(\models \phi \rightarrow \psi\) тогда и только тогда, когда \(\phi \models{\psi}\).

        По определению \(\phi \rightarrow \psi\) -- краткая запись \(\neg \phi \lor \psi\).
        Если \(\phi \models{\psi}\), тогда если \(M_i\llbracket \phi \rrbracket = 1\), то и \(M_i\llbracket \psi \rrbracket = 1\).

        Рассмотрим два случая.
        \begin{enumerate}
            \item
                \(M_i\llbracket \phi \rrbracket = 0\).
                Получаем \(M_i\llbracket \neg \phi \lor \psi \rrbracket = MAX(1, M_i\llbracket \psi \rrbracket) = 1\)
            \item
                \(M_i\llbracket \phi \rrbracket = 1\).
                Тогда получаем \(M_i\llbracket \neg \phi \lor \psi \rrbracket = MAX(0, M_i\llbracket \psi \rrbracket) = MAX(0, 1) = 1\)
        \end{enumerate}

        Во второй интерпретации формула будет выполняться только при условии \(\phi \models \psi\), ч.т.д.
    \item
        Доказать, что \(\models \phi \leftrightarrow \psi\) тогда и только тогда, когда \(\phi \sim \psi\).

        По определению \(\phi \leftrightarrow \psi\) -- краткая запись \((\phi \land \psi) \lor (\neg \phi \land \neg \psi)\).
        Если \(\phi \sim \psi\), тогда если \(M_i\llbracket \phi \rrbracket = 1\),
        то \(
        M_i\llbracket \psi \rrbracket = 1\) и если
        \(M_j\llbracket \psi \rrbracket = 1\), то
        \(M_j\llbracket \phi \rrbracket = 1
        \)

        Рассмотрим два случая.
        \begin{enumerate}
            \item
                \(M_i\llbracket \phi \rrbracket = 0\). Получаем:
                \[
                    M_i\llbracket (\phi \land \psi) \lor (\neg \phi \land \neg \psi) \rrbracket
                    = MAX(MIN(0, M_i\llbracket\psi\rrbracket), MIN(1, 1))
                    = MAX(0, 1)
                    = 1
                \]
                Мы принимаем \(M_i\llbracket\neg\psi\rrbracket\) равным 1 т.к. 
                если бы \(M_i\llbracket\psi\rrbracket = 1\), то и \(M_i\llbracket\phi\rrbracket = 1\), 
                что противоречит условию рассматриваемого случая.
            \item
                \(M_i\llbracket \phi \rrbracket = 1\). Получаем:
                \[
                    M_i\llbracket (\phi \land \psi) \lor (\neg \phi \land \neg \psi) \rrbracket
                    = MAX(MIN(1, 1), MIN(0, 0))
                    = MAX(1, 0)
                    = 1
                \]
        \end{enumerate}

        Первый случай выполняется только при \(\psi \models \phi\).
        В то время как второй случай выполняется в силу \(\phi \models \psi\).
        Т.е. оба случая выполняются тогда и только тогда, когда \(\phi \sim \psi\), ч.т.д.
\end{enumerate}

\section*{Задание 2}
\begin{enumerate}
    \item
        Докажем \(\neg \neg \phi \sim \phi\). Данная запись означает, 
        что значения формул \(\neg \neg \phi\) и \(\phi\) совпадают при любой интерпретации 
        т.е. \(M\llbracket \neg \neg \phi\rrbracket = M\llbracket\phi\rrbracket\).

        Упростим левую часть выражения:
        \[
            M\llbracket\neg \neg \phi \rrbracket
            = 1 - M\llbracket \neg \phi \rrbracket
            = 1 - (1 - M\llbracket\phi\rrbracket)
            = M\llbracket\phi\rrbracket
        \]
        Ч.т.д.
    \item
        Докажем \(\models \phi \lor \neg \phi\).
        \[
            M\llbracket \phi \lor \neg \phi \rrbracket
            = MAX(M\llbracket \phi \rrbracket, 1 - M\llbracket \phi \rrbracket)
        \]
        Получаем, что мы выбираем максимум из 1 и 0 независимо от интерпретации формулы. 
        Т.е. выражение \(\phi \lor \neg \phi\) общезначимо. Ч.т.д.
    \item
        Докажем \(\phi \land (\psi \lor \eta) \sim (\phi \land \psi) \lor (\phi \land \eta)\).
        Если верно исходное утверждение, то верно и:
        \[M \llbracket \phi \land (\psi \lor \eta) \rrbracket = M \llbracket (\phi \land \psi) \lor (\phi \land \eta) \rrbracket\]
        Упростим обе части равенства.
        \[
            M \llbracket \phi \land (\psi \lor \eta) \rrbracket 
            = MIN(M \llbracket \phi \rrbracket, MAX(M \llbracket \psi \rrbracket, M \llbracket \eta \rrbracket))
        \]
        \[
            M \llbracket (\phi \land \psi) \lor (\phi \land \eta) \rrbracket
            = MAX(MIN(M \llbracket \phi \rrbracket, M \llbracket \psi \rrbracket), MIN(M \llbracket \phi \rrbracket, M \llbracket \eta \rrbracket))
        \]
        Заметим, что при \(M \llbracket \phi \rrbracket = 0\), обе формулы обращаются в 0, 
        а при \(M \llbracket \phi \rrbracket = 1\) в \(MAX(M \llbracket \psi \rrbracket, M \llbracket \eta \rrbracket)\)
        Следовательно, значения этих формул эквиваленты в любой из интерпретаций, ч.т.д.
    \item 
        Докажем \(\phi \lor (\psi \land \eta) \sim (\phi \lor \psi) \land (\phi \lor \eta)\).
        Если верно исходное утверждение, то верно и:
        \[M \llbracket \phi \lor (\psi \land \eta) \rrbracket = M \llbracket (\phi \lor \psi) \land (\phi \lor \eta) \rrbracket\]
        Упростим обе части равенства.
        \[
            M \llbracket \phi \lor (\psi \land \eta) \rrbracket 
            = MAX(M \llbracket \phi \rrbracket, MIN(M \llbracket \psi \rrbracket, M \llbracket \eta \rrbracket))
        \]
        \[
            M \llbracket (\phi \lor \psi) \land (\phi \lor \eta) \rrbracket
            = MIN(MAX(M \llbracket \phi \rrbracket, M \llbracket \psi \rrbracket), MAX(M \llbracket \phi \rrbracket, M \llbracket \eta \rrbracket))
        \]
        Заметим, что при \(M \llbracket \phi \rrbracket = 1\), обе формулы обращаются в 1, 
        а при \(M \llbracket \phi \rrbracket = 0\) в \(MIN(M \llbracket \psi \rrbracket, M \llbracket \eta \rrbracket)\)
        Следовательно, значения этих формул эквиваленты в любой из интерпретаций, ч.т.д.
    \item
        Докажем \(\phi \lor (\phi \land \psi) \sim \phi\).
        Если верно исходное утверждение, то верно и:
        \[M \llbracket \phi \lor (\phi \land \psi) \rrbracket = M \llbracket \phi \rrbracket\]
        Упростим левую часть равенства.
        \[
            M \llbracket \phi \lor (\phi \land \psi) \rrbracket 
            = MAX(M \llbracket \phi \rrbracket, MIN(M \llbracket \phi \rrbracket, M \llbracket \psi \rrbracket))
        \]
        Заметим, что при \(M \llbracket \phi \rrbracket = 1\), формула обращается в 1, 
        а при \(M \llbracket \phi \rrbracket = 0\) в 0.
        Следовательно, 
        \(M \llbracket \phi \lor (\phi \land \psi) \rrbracket 
        = M \llbracket \phi \rrbracket\), ч.т.д.
    \item
        Докажем \(\phi \land (\phi \lor \psi) \sim \phi\).
        Если верно исходное утверждение, то верно и:
        \[M \llbracket \phi \land (\phi \lor \psi) \rrbracket = M \llbracket \phi \rrbracket\]
        Упростим левую часть равенства.
        \[
            M \llbracket \phi \land (\phi \lor \psi) \rrbracket 
            = MIN(M \llbracket \phi \rrbracket, MAX(M \llbracket \phi \rrbracket, M \llbracket \psi \rrbracket))
        \]
        Заметим, что при \(M \llbracket \phi \rrbracket = 1\), формула обращается в 1, 
        а при \(M \llbracket \phi \rrbracket = 0\) в 0.
        Следовательно, 
        \(M \llbracket \phi \land (\phi \lor \psi) \rrbracket 
        = M \llbracket \phi \rrbracket\), ч.т.д.
    \item
        Докажем \(\neg (\phi \land \psi) \sim \neg \phi \lor \neg \psi\).
        Если верно исходное утверждение, то верно и:
        \[M \llbracket \neg (\phi \land \psi) \rrbracket 
        = M \llbracket \neg \phi \lor \neg \psi \rrbracket\]
        Упростим обе части равенства.
        \[
            M \llbracket \neg (\phi \land \psi) \rrbracket 
            = 1 - MIN(M \llbracket \phi \rrbracket, M \llbracket \psi \rrbracket)
        \]
        \[
            M \llbracket \neg \phi \lor \neg \psi \rrbracket
            = MAX(1 - M \llbracket \phi \rrbracket, 1 - M \llbracket \psi \rrbracket)
        \]
        Заметим, что при \(M \llbracket \phi \rrbracket = 0\), обе формулы обращаются в 1, 
        а при \(M \llbracket \phi \rrbracket = 1\) в \(1 - M \llbracket \psi \rrbracket\).
        Следовательно, значения этих формул эквиваленты в любой из интерпретаций, ч.т.д.
    \item
        Докажем \(\neg (\phi \lor \psi) \sim \neg \phi \land \neg \psi\).
        Если верно исходное утверждение, то верно и:
        \[M \llbracket \neg (\phi \lor \psi) \rrbracket 
        = M \llbracket \neg \phi \land \neg \psi \rrbracket\]
        Упростим обе части равенства.
        \[
            M \llbracket \neg (\phi \lor \psi) \rrbracket 
            = 1 - MAX(M \llbracket \phi \rrbracket, M \llbracket \psi \rrbracket)
        \]
        \[
            M \llbracket \neg \phi \land \neg \psi \rrbracket
            = MIN(1 - M \llbracket \phi \rrbracket, 1 - M \llbracket \psi \rrbracket)
        \]
        Заметим, что при \(M \llbracket \phi \rrbracket = 0\), 
        обе формулы обращаются в \(1 - M \llbracket \psi \rrbracket\), 
        а при \(M \llbracket \phi \rrbracket = 1\) в 0.
        Следовательно, значения этих формул эквиваленты в любой из интерпретаций, ч.т.д.
\end{enumerate}

\section*{Задание 3}
\begin{enumerate}
    \item
        Определим общезначима ли формула \(
            (p \rightarrow q) 
            \leftrightarrow (\neg q \rightarrow \neg p)
        \).
        Раскроем макрос импликации согласно определению. Получаем: \(
            (\neg p \lor q) 
            \leftrightarrow (q \lor \neg p)
        \). Заметим, что левая часть выражения равна правой с 
        точностью до перестановки литералов т.к. операция \(\lor\) коммутативна. 
        Заменим для удобства (\(p \lor q)\) на \(\alpha\).
        Тогда по определению раскрываем макрос эквивалентности: \(
            \alpha \leftrightarrow \alpha 
            = (\alpha \land \alpha) \lor (\neg \alpha \land \neg \alpha)
        \). Следовательно, \(
            M\llbracket (\alpha \land \alpha) \lor (\neg \alpha \land \neg \alpha) \rrbracket
            = MAX(MIN(M\llbracket \alpha \rrbracket, M\llbracket \alpha \rrbracket), MIN(1 - M\llbracket \alpha \rrbracket, 1 - M\llbracket \alpha \rrbracket))
            = MAX(M\llbracket \alpha \rrbracket, 1 - M\llbracket \alpha \rrbracket) = 1 
        \). Т.е. исследуемая формула независимо от интерпретации всегда обращается в 1, 
        что по определению делает её общезначимой, ч.т.д.
    \item
        Определим общезначимость формулы \(
            (p \rightarrow (q \rightarrow r)) 
            \leftrightarrow (\neg r \rightarrow (\neg q \rightarrow \neg p))
        \). Раскроем макрос импликации согласно определению. Получаем: \(
            (\neg p \lor \neg q \lor r) \leftrightarrow (r \lor q \lor \neg p)
        \). Заменим для удобства (\(\neg p \lor r)\) на \(\alpha\).
        Получаем: \((\neg q \lor \alpha) \leftrightarrow (q \lor \alpha)\).
        Можно заметить, что если \(M \llbracket \alpha \rrbracket = 0\), 
        то выражение не выполнится ни в одной из возможных интерпретаций. Если же 
        \(M \llbracket \alpha \rrbracket = 1\), то выражение выполнится независимо 
        от значения \(M \llbracket q \rrbracket\).
        
        Подберём соответствующие интерпретации \(M_1\) и \(M_2\), такие что: 
        \(M_1 \models \phi\) и \(M_2 \nvDash \phi\), где \(\phi\) исходная формула.
        \[M_1 = [p \mapsto 0, r \mapsto 0, q \mapsto 0]\]
        \[M_2 = [p \mapsto 1, r \mapsto 0, q \mapsto 0]\]
        Следовательно, исходная формула выполнимая и необщезначимая.
\end{enumerate}

\section*{Задание 4}
Приведём формулу \(\neg(\neg(p \land q) \rightarrow \neg r)\) в КНФ и ДНФ.

Воспользуемся законами, доказанными во втором задании.

\[
    \neg(\neg(p \land q) \rightarrow \neg r) 
    = \neg((p \land q) \lor \neg r)
    = \neg (p \land q) \land r
    = (\neg p \lor \neg q) \land r 
    = (r \land \neg p) \lor (r \land \neg q)
\]

КНФ: \((\neg p \lor \neg q) \land r\)

ДНФ: \((r \land \neg p) \lor (r \land \neg q)\)

Т.к. в ННФ входят только конъюнкции, дизъюнкции и литералы, то обе полученные формулы находятся в ННФ.

\section*{Задание 5}
Докажем теорему о монотонности ННФ.
Воспользуемся индукцией по построению формулы.

База: формула \(\phi\), состоящая из одного литерала \(p_1\), такая что
\(I \models \phi\). Очевидно, что если выполняется условие 
\(positive(I, \phi) \subseteq positive(I', \phi)\), то справедливо и \(I' \models \phi\).
Т.к. расширяя множество positive мы не изменяем истинности существующего литерала.

Переход: допустим, что: 
\[\phi_n = \phi_{n-1} \odot p\] 
\[I \models \phi_{n-1}\]
\[I' \models \phi_{n-1}\]
\[I \models \phi_{n}\]
\[I' \llbracket p \rrbracket = 1 \]

Тогда возможны два случая:
\begin{enumerate}
    \item \(I \llbracket p \rrbracket = 1 \). 
    Вместо \(\odot\) может стоять любой оператор: \(\lor\) или \(\land\).
    Но т.к. \(I' \models \phi_{n-1}\) и \(I' \models p\), то независимо от оператора
    верно \(I' \models \phi_{n}\).

    \item \(I \llbracket p \rrbracket = 0 \). 
    Т.к. \(I \models \phi_n\) мы можем исключить из рассмотрения оператор \(\land\).
    А т.к. \(I' \models \phi_{n-1}\), то верно и 
    \(I' \models \phi_{n-1} \lor p\) т.е. верно 
    \(I' \models \phi_{n}\)
\end{enumerate}

Ч.т.д.

\section*{Задание 6}
Докажем теорему о том, что для всякой булевой функции n аргументов 
найдётся формула логики высказываний \(\phi\) от n \(p_1,...,p_n\) переменных, что для всех интерпретаций M.
имеет место
\[M \llbracket \phi \rrbracket = f(M(p_1), ..., M(p_n))\]

Воспользуемся индукцией по построению формулы, где \(n\) - число переменных.

База: \(n = 1\).
Мы можем построить формулу тождественной функции: \(p_1\).

Переход: 
Допустим, что \(\phi_n\) и \(\psi_m\) -- формулы логики высказываний n и m переменных, 
соответствующие булевым функциям n и m аргументов.

\begin{enumerate}
    \item 
        Чтобы получить отрицание функции мы можем записать 
        отрицание формулы следующим образом: \(\neg \phi_n\)
    \item 
        Для записи булевого сложения двух булевых функций можно воспользоваться
        следующей записью: \((\phi_n) \lor (\psi_m)\).
    \item 
        Для записи булевого умножения двух булевых функций можно воспользоваться
        следующей записью: \((\phi_n) \land (\psi_m)\).
\end{enumerate}

Таким образом, мы можем выразить любую булеву функцию 
с помощью формулы логики высказываний. Ч.т.д.
\section*{Задание 7}
\begin{enumerate}
    \item
    Докажем, что для любых интерпретаций M и N верно, что \(M^{-1}(1) \subseteq N^{-1}(1)\)
    тогда и только тогда, когда всякое позитивное высказывание, истинное в M, истинно и в N.

    От противного. Допустим, что любое позитивное высказывание \(\phi_i\) верное в 
    \(M\) верно и в \(N\) и \(M^{-1}(1) \nsubseteq N^{-1}(1)\).

    Тогда если \(p_i \in M^{-1}(1)\) и \(p_i \notin I^{-1}(1)\), то формула из одного литерала \(p_i\) 
    выполнима в \(M\), но не в \(N\), что противоречит исходному условию.

    Пусть \(M^{-1}(1) \subseteq N^{-1}(1)\), \(M \models \phi\), \(N \nvDash \phi\). 
    Что противоречит монотонности позитивных формул. Доказательство аналогично заданию 5.

    Таким образом, \(M^{-1}(1) \subseteq N^{-1}(1)\) тогда и только тогда, 
    когда всякое позитивное высказывание, истинное в M, истинно и в N.    
\end{enumerate}

\section*{Задание 8}
Пусть \(\phi\) -- выполнимая формула над счётном множеством переменных, а 
\(M_1\) -- некая модель \(\phi\). Любую переменную, не входящую в в формулу \(\phi\), 
мы можем отображать как на 0, так и на 1. Следовательно, множество 
моделей формулы \(\phi\) больше мощности счётного множества переменных 
т.е. имеет мощность континуума, ч.т.д.  

\section*{Задание 9}
Опишем исчисление, в котором выводимыми являются в точности все формулы логики
высказываний.

Определим алфавит 
\(\Sigma = \{\top, \bot\}\ \cup\ \{\neg, \lor, \land, (, )\}\ \cup\ \mathcal{V}\)
и язык
\(\mathcal{L} = \Sigma^*\), где \(\mathcal{V}\) -- множество переменных.

Множество аксиом \(\mathcal{A\chi} = \{\top, \bot\}\ \cup\ \mathcal{V}\).

Определим правила вывода:
\begin{enumerate}
    \item 
        \begin{prooftree}
            \AxiomC{\(\phi\)}
            \UnaryInfC{\((\phi)\)}
        \end{prooftree}
    \item 
        \begin{prooftree}
            \AxiomC{\(\phi\)}
            \UnaryInfC{\(\neg \phi\)}
        \end{prooftree}
    \item 
        \begin{prooftree}
            \AxiomC{\(\phi\)}
            \AxiomC{\(\psi\)}
            \BinaryInfC{\(\phi \land \psi\)}
        \end{prooftree}
    \item
        \begin{prooftree}
            \AxiomC{\(\phi\)}
            \AxiomC{\(\psi\)}
            \BinaryInfC{\(\phi \lor \psi\)}
        \end{prooftree}
\end{enumerate}

\section*{Задание 10}
Опровергнем множество дизъюнктов методом резолюций.
\begin{enumerate}
    \item \(x_1\)
    \item \(\neg x_1 \lor \neg x_2 \lor x_3\)
    \item \(x_2 \lor \neg x_4\)
    \item \(x_4\)
    \item \(\neg x_5 \lor \neg x_3\)
    \item \(x_5\)
\end{enumerate}

\begin{prooftree}
    \AxiomC{\(\neg x_5 \lor \neg x_3\)}
    \AxiomC{\(x_5\)}
    \BinaryInfC{\(\neg x_3\)}
    \AxiomC{\(\neg x_1 \lor \neg x_2 \lor x_3\)}
    \BinaryInfC{\(\neg x_1 \lor \neg x_2\)}
    \AxiomC{\(x_1\)}
    \BinaryInfC{\(\neg x_2\)}
    \AxiomC{\(x_2 \lor \neg x_4\)}
    \BinaryInfC{\(\neg x_4\)}
    \AxiomC{\(x_4\)}
    \BinaryInfC{\boxed{}}
    \end{prooftree}
\end{document}